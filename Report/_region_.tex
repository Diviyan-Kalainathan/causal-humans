\message{ !name(main.tex)}%%%%%%%%%%%%%%%%%%%%%%%%%%%%%%%%%%%%%%%%%
%%Template by Mathias Legrand
%  My documentation report
%  Objetive: Explain what I did and how, so someone can continue with the investigation
%
% Important note:
% Chapter heading images should have a 2:1 width:height ratio,
% e.g. 920px width and 460px height.
%
%%%%%%%%%%%%%%%%%%%%%%%%%%%%%%%%%%%%%%%%%

%----------------------------------------------------------------------------------------
%	PACKAGES AND OTHER DOCUMENT CONFIGURATIONS
%----------------------------------------------------------------------------------------

\documentclass[11pt,fleqn,openany,frenchb]{book} % Default font size and left-justified equations
\usepackage[top=3cm,bottom=3cm,left=3.2cm,right=3.2cm,headsep=10pt,a4paper]{geometry} % Page margins
\usepackage{xcolor} % Required for specifying colors by name
\definecolor{ocre}{RGB}{20,60,120} % Define the orange color used for highlighting throughout the book
\definecolor{ocredark}{RGB}{5,20,60} 
% Font Settings
\usepackage{avant} % Use the Avantgarde font for headings
%\usepackage{times} % Use the Times font for headings
\usepackage{mathptmx} % Use the Adobe Times Roman as the default text font together with math symbols from the Sym­bol, Chancery and Com­puter Modern fonts
%\usepackage{hyperref}
\usepackage{microtype} % Slightly tweak font spacing for aesthetics
\usepackage[utf8]{inputenc} % Required for including letters with accents
\usepackage[T1]{fontenc} % Use 8-bit encoding that has 256 glyphs
\usepackage[french,onelanguage]{algorithm2e}

\usepackage{eurosym}%pour le symbole €
\usepackage{subfig}
\def\INDEP{{\sc Indep.}}
\def\SANTE{{\sc Sante}}
\def\OUVR{{\sc Ouvriers}}

\def\CSPPPr{{\sc CSP+Privé}}
\def\CSPPPu{{\sc CSP+Public}}
\def\IMM{{\sc Immigr.}}
\def\SERV{{\sc ServPart}}
\def\ACC{{\sc Accid.}}

%%%%%%%%%%%
\def\MALH{{\sc Malh.}}
\def\RAS{{\sc RAS}}
\def\HEUR{{\sc Heur.}}
\def\ENV{{\sc Stress}}
\def\GLOB{{\sc Chgts}}

%%%%%%%%%%%%%%%
\def\BSERV{{\sc ServPersonnes}}
\def\PRESS{{\sc Presse}}
\def\TRAD{{\sc Industrie}}
\def\HT{{\sc High-Tech}}


% Bibliography
\PassOptionsToPackage{hyphens}{url}
\usepackage{hyperref}
\hypersetup{hidelinks,backref=true,pagebackref=true,hyperindex=true,colorlinks=false,breaklinks=true,urlcolor= ocre}%,%bookmarks=true,bookmarksopen=false,pdftitle={Title},pdfauthor={Author}}
%\usepackage[style=alphabetic,sorting=nyt,sortcites=true,autopunct=true,babel=hyphen,hyperref=true,abbreviate=false,backref=true,backend=biber]{biblatex}
%\addbibresource{bibliography.bib} % BibTeX bibliography file
%\defbibheading{bibempty}{}
%

\input{structure} % Insert the commands.tex file which contains the majority of the structure behind the template
%\usepackage{xcolor}
\usepackage[Sonny]{fncychap}
  % defaults:
  \ChNameVar{\Large\sf\color{ocredark}}
  \ChNumVar{\Huge\color{ocredark}}
  % custom:
  \ChTitleVar{\Large\scshape\color{ocre}}
\begin{document}

\message{ !name(main.tex) !offset(99) }
\section{Démarche}
Après une première phase de prétraitement des données, l'étude est
divisée en deux : on effectuera une analyse descriptive des données
poussée, avant d'effectuer une analyse causale, où nous chercherons à
déterminer les relations d'implication entre les variables du
questionnaire.\par

\subsection{Détermination des profils types}
Les données pré-traitées comportent un nombre important de variables,
rendant l'étude complexe : regrouper les individus pour former des
profils types nécessite de prendre en compte toutes les variables. La
solution employée est l'analyse en composantes principales, qui permet
de remédier à la redondance des variables, pour définir un petit
nombre d'axes (variables agrégées, définie par une somme pondérée des
variables initiales) capturant la variabilité des
données. L'interprétation d'un axe se fait en considérant les
variables initiales les plus importantes (valeurs absolues des poids
les plus élevés). \par

Dans l'espace des axes, chaque individu est un vecteur de
$\mathbb{R}^d$. On utilise la catégorisation (clustering) pour
identifier les sous-groupes de données homogènes ; l'algorithme
employé est un K-means++ \cite{arthur2007k}. Avant d'analyser les
clusters, on s'assure de leur stabilité selon les critères définis par
\cite{meilua2006uniqueness}. \par

Chaque cluster est interprété par ses variables significatives au sens
de la mesure statistique valeur-test \cite{lebart2006statistique}~;
formellement, une variable est significative pour un cluster lorsque
sa valeur moyenne sur ce cluster est significativement distincte de la
valeur moyenne sur l'ensemble des données (compte tenu de la taille du
cluster). Après avoir établi des clusters sur les variables de
situation et de ressenti (respectivement sur les variables objectives
et subjectives), il s'agit d'analyser comment évoluent les groupes à
travers des variables choisies, telles que le revenu ou le score de
bien-être défini par
l'OMS\footnote{\href{http://www.euro.who.int/fr/publications/abstracts/measurement-of-and-target-setting-for-well-being-an-initiative-by-the-who-regional-office-for-europe}{cf
    www.euro.who.int.}} ; mais aussi étudier le croisement des
populations entre les groupes objectifs et subjectifs est une analyse
qui permet mettre en évidence le lien entre la situation de l'enquêté
et son ressenti de sa situation. La méthodologie est illustrée à la
Fig.\ref{fig:metho}.\par

\begin{figure}[!h]
  \centering
  \includegraphics[height=13cm]{schema_methode_fr.pdf}
  \caption{Méthodologie de l'analyse descriptive des données}
  \label{fig:metho}
\end{figure}

\subsection{Causalité}
La deuxième partie de l'étude consiste à approfondir l'étude en
étudiant la causalité au sens de \cite{granger1969causality} ; la
causalité inclut plus d'informations qu'une simple corrélation, par la
présence d'une hiérarchie entre les variables reliées causalement. En
effet, la présence d'une corrélation traduit juste la "ressemblance
entre deux courbes", et ne permet pas de conclure sur l'existence d'un
réel lien entre les deux variables\footnote{Par exemple, la
  corrélation entre le nombre de pirates en activité et le
  réchauffement climatique est importante alors que ces deux variables
  ne sont pas directement liées causalement.}. L'étude de la
causalité, se basant sur des techniques complexes et variées, entre
prédiction par machine learning et inférence par l'étude des
distributions de probabilités permettent de déterminer la présence ou
non d'une relation de causalité, mais aussi du sens de cette
relation. Ainsi, on aura pour but de construire le graphe le plus
complet et le plus fiable des variables et de leurs liens causaux,
afin de comprendre les phénomènes moteurs dans le questionnaire et
dans l'étude du bien-être au travail. Le but étant de pouvoir faire
des recommendations aux managers afin d'améliorer la qualité de vie au
travail, les enjeux que représentent cette étude sont mis en valeur
par le fait que la causalité à l'aide d'analyse de données n'a pas été
étudiée jusqu'à aujourd'hui en sociologie. \par % a compléter

% ----------------------------------------------------------------------------------------
% CHAPTER 1
% ----------------------------------------------------------------------------------------
% \chapterimage{blank_chap2.pdf}
\message{ !name(main.tex) !offset(721) }

\end{document}
